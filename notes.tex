\documentclass{article}
\usepackage{amsmath}
\renewcommand{\eqref}[1]{Eq.~(\ref{#1})}
\renewcommand{\exp}[1]{\ensuremath{\mathrm{e}^{#1}}}

\title{Notes on inversion algorithms}
\author {Jonathan G. Underwood}

\begin{document}
\maketitle
\tableofcontents

\section{Coordinate systems}
We adopt the convention that the observed image $F(x,z)$ lies in the
$xz$-plane, with $z$ designated as the axis of cylindrical symmetry. We also
wish to express the image in polar coordinates, $F(R, \Theta)$ such that
\begin{gather}
  z=R\cos\Theta\\
  x=R\sin\Theta\\
  R^2=x^2+z^2
\end{gather}
The distribution of photofragments is most naturally expressed in polar
coordinates, $f(r, \theta, \phi)\equiv f(r, \theta)$. We assume a cylindrical
distribution of fragments. We choose a system of coordinates such that the $x$
and $z$ directions are coincident to that used for the image plane.
\begin{gather}
  z=r\cos\theta\\
  x=r\sin\theta\cos\phi\\
  y=r\sin\theta\sin\phi\\
  r^2=x^2+y^2+z^2
\end{gather}
It is useful to define the quantity $\rho$ representing the distance from the
tip of the vector $\mathbf{r}$ to the $z$-axis
\begin{gather}
  \rho=r\sin\theta\\
  \rho^2=x^2+y^2
\end{gather}
We also note the following relationships
\begin{gather}
  r^2=R^2+y^2\\
  \rho^2=r^2-z^2\\
  R=\frac{r\cos\theta}{\cos\Theta}\\
  \tan\Theta=\frac{x}{z}=\tan\theta\cos\phi
\end{gather}

\section{Forward Abel integral expressed in Cartesian coordinates}
We can express the image $F(x,z)$ in terms of the distribution expressed in
cylindrical polar coordinates $f(\rho,z\, \phi)\equiv f(\rho, z)$ as 
\begin{equation}
  F(x,z)=\int_{-\infty}^{\infty}
  f(\rho, z)\;\mathrm{d}y.
\end{equation}
Since $y=\sqrt{\rho^2-x^2}$,
\begin{equation}
  \frac{\mathrm{d}y}{\mathrm{d}\rho}
  =\frac{\rho}{\sqrt{\rho^2-x^2}}
\end{equation}
and so
\begin{equation}
  \label{eq:abel_cartesian}
  F(x,z)=2\int_{|x|}^{\infty}
  \frac{\rho f(\rho, z)}{\sqrt{\rho^2-x^2}}
  \;\mathrm{d}\rho.
\end{equation}

\section{Forward Abel integral expressed in spherical polar coordinates} 
We would like to express the observed image $F(R, \Theta)$ in terms of the
distribution $f(r, \theta)$. The Abel integral is
\begin{equation}
  F(R, \Theta)=2\int_{|x|}^\infty
  \frac{\rho f(r, \theta)}{\sqrt{\rho^2-x^2}}\;\mathrm{d}\rho.
\end{equation}
with $x=R\sin\Theta$ and $\rho=r\sin\theta$. Since $\rho=\sqrt{r^2-z^2}$,
\begin{equation}
  \frac{\mathrm{d}\rho}{\mathrm{d}r}=
  \frac{r}{\sqrt{r^2-z^2}},
\end{equation}
such that
\begin{equation}
  \rho\;\mathrm{d}\rho=r\;\mathrm{d}r
\end{equation}
and since $r=\sqrt{\rho^2+z^2}$
\begin{equation}
  F(R, \Theta)=
  2\int_{\sqrt{|x|^2+z^2}}^\infty
  \frac{rf(r, \theta)}{\sqrt{r^2-z^2-x^2}}\;\mathrm{d}r
\end{equation}
which can be re-expressed as
\begin{equation}
  \label{eq:abel_polar}
  F(R, \Theta)=
  2\int_{R}^\infty
  \frac{rf(r, \theta)}{\sqrt{r^2-R^2}}\;\mathrm{d}r.
\end{equation}
with 
\begin{equation}
  \theta=\arccos\left(
    \frac{R}{r}\cos\Theta
  \right)
\end{equation}
This equation is analogous to \eqref{eq:abel_cartesian} with the equivalence
$r\leftrightarrow\rho$ and $R\leftrightarrow x$. However, the equivalence
$\Theta\leftrightarrow z$ is not quite so straightforward.

It is readily seen that the inverse transform of \eqref{eq:abel_polar}
diverges as $r\rightarrow0$. This has the consequence of placing inversion
noise at the centre of the inverted image, rather than along the $z$-axis
(i.e. as $x\rightarrow0$) as is the case of the inverse of
\eqref{eq:abel_cartesian}.

\section{Implementation notes for pbasex}
We use a slightly modified implementation of pbasex, using
\eqref{eq:abel_polar}. Following GNP we expand the photofragment distribution
as 
\begin{equation}
  f(r,\theta)=
  \sum_{k=0}^{k_\mathrm{max}}
  \sum_{l=0}^{l_\mathrm{max}}
  c_{kl}f_{kl}(r,\theta)
\end{equation}
where
\begin{equation}
  f_{kl}(r,\theta)=
  \exp{-(r-r_k)^2/\sigma}P_l(\cos\theta).
\end{equation}
The image can then be expressed as
\begin{equation}
  F(R, \Theta)=
  \sum_{k=0}^{k_\mathrm{max}}
  \sum_{l=0}^{l_\mathrm{max}}
  c_{kl}g_{kl}(R,\Theta)
\end{equation}
with
\begin{equation}
  g_{kl}(R,\Theta)=
  2\int_{R}^\infty
  \frac{rf_{kl}(r, \theta)}{\sqrt{r^2-R^2}}\;\mathrm{d}r.
\end{equation}

The observed image is actually sampled on a grid of cartesian pixels which is
the converted to a regular grid of $(R, \Theta)$ pixels.

\begin{equation}
  F_{ij}(R, \Theta)=
  \sum_{k=0}^{k_\mathrm{max}}
  \sum_{l=0}^{l_\mathrm{max}}
  c_{kl}g_{ij;kl}(R,\Theta)
\end{equation}
where
\begin{equation}
  g_{ij;kl}(R,\Theta)=
  2
  \int_{i\Delta_R}^{(i+1)\Delta_R}\mathrm{d}R
  \int_{j\Delta_\Theta}^{(j+1)\Delta_\Theta}\sin\Theta\;\mathrm{d}\Theta
  \int_{R}^\infty
  \frac{rf_{kl}(r, \theta)}{\sqrt{r^2-R^2}}\;\mathrm{d}r.
\end{equation}

\end{document}
%%% Local Variables: 
%%% mode: latex
%%% TeX-master: t
%%% End: 
